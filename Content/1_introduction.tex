\section{Area of interest}
The subject of parallel computing is one that has become highly relevant and important in recent years.
Moore's law, the observed pattern that the number of transistors in a dense integrated circuit doubles approximately every two
years \cite{moore_1998_cramming_cmcoic},
has lost its relevance. The increased processor clock speed that the doubling in processors implies is no longer present because of
overheating issues \cite[Ch. 1]{herlihy_2012_art_taomprr}. Because of this, manufacturers of processors now have
largely turned to \emph{multicore} processors. In a multicore architecture, several cores which work as individual processors execute
code simultaneously. Using this type of architecture to work on a single task to increase performance is known as \emph{parallelism}.

Efforts to exploit parallelism automatically from a program have been made; however, the benefits of these have reached their
limit \cite{mccool_2012_structured_spppfec}. In order to fully utilize the increase in performance that multicore architectures promise, 
programmers today must instead turn to explicit parallel programming.

Python is one of world's most popular programming languages \cite{krill_2015_python_psnhilp}. It is used extensively both at schools and
in the industry, and its benefits include expressiveness, portability, and the fact that it is easy to learn. Python has support for
parallel programming, although it has caveats and overheads mainly as a result of a concurrency-hampering mechanism called the
\emph{Global Interpreter Lock} \cite{beazley_150745UTC_introduction_aitpc}.

This thesis concerns a combination of the areas mentioned above: parallel computing using Python.

\section{Trioptima}

\section{Problem statement}

\section{Research question}

\section{Objective}

\section{Motivation}

\section{Delimitations}

