In this chapter, an introduction to the parallel computing problem domain and the specific problem of dataset transformation are
described in order to give the reader an initial view of what this thesis entails.

\section{Dataset transformation} \label{dataset_standardization}
In financial applications concerning trading, it is common for customers to upload datasets containing several rows describing trades, which may be in different formats.
One such application is triResolve, an application maintained by TriOptima, where this thesis is conducted.  In triResolve, customers resolve trade disputes in the
OTC derivatives market, which may arise due to for example differences in valuation methods. The triResolve service is built in Python, To be able to resolve
the trade disputes, customers upload the previously mentioned datasets to the service.

The datasets need to be processed in order to transform them into a standard format which makes comparisons between data from different customers possible.
In some cases, the size of the dataset is large enough that this transformation is slow. Out of the possible techniques for optimizing the transformation code,
this thesis will focus on parallelization. Since Python is the language used in triResolve, the parallelization of the existing program will be implemented using
parallel programming tools available in the language. 

When parallelizing a program, the workload is divided amongst multiple cores of a system, which execute the program in parallel.
For the dataset transformation problem, this means dividing the dataset, conceivably into chunks of rows, and performing the transformation of each of these chunks
on separate cores.

This thesis present the challenges associated with this parallelization problem, and how to solve them.

%In some cases, the size of the dataset is large enough that this transformation is slow, and could conceivably be sped up through
%the use of parallelization. The sizes are aptly measured in number of rows, and range between 2 rows to about 1490000 rows. The time
%it takes to process the datasets range between 0.06 seconds and 15200 seconds 2 (4.2 hours).

%\subsection{Transformation with constraints}
The datasets are associated with a file format. The format specifies a set of rules, known as filters, which at times enforce
implicit constraints on the processing order in the file when performing the transformation. This thesis aims to identify these
constraints, which may affect how parallelizable a dataset is, and find a suitable parallelization strategy. Another aim is to identify the impact dataset size has on
any potential speedup. In addition, the effect that using the Python \code{multiprocessing} module and its process-over-thread with message passing
approach affects implementation and performance will be investigated.

%reduce the possible benefits of parallelization as they enforce
%inherently serial parts of the transformation program.  Since the size of the datasets as well as the type and number of their
%associated filters varies, it is plausible that the benefits of parallelization will differ significantly between different datasets.
%An overhead is associated with creating new threads or processes.  This overhead is increased in Python as the data shared between processes needs to serialized.
%Therefore, it is possible that parallelization of datasets will result in slower execution in some cases. Consequently, it is
%interesting to find the combinations of dataset sizes, as well as their filters, that result in beneficial parallelization, and which do not.
%Additionally, the complex nature of the system makes the implementation of the parallelization an interesting problem.

\section{Parallel computing}
The subject of parallel computing is one that has become highly relevant in recent years.
Moore's law, the observed pattern that the number of transistors in a dense integrated circuit doubles approximately every two
years \cite{moore_1998_cramming_cmcoic},
has lost its relevance. The increased processor clock speed that the doubling in processors implies is no longer present because of
overheating issues \cite[p. 1]{herlihy_2012_art_taomprr}. Because of this, manufacturers of processors now have
largely turned to \emph{multicore} processors. In a multicore architecture, several cores which work as individual processors execute
code simultaneously. Using this type of architecture to work on a single task to increase performance is known as \emph{parallelism}.

Efforts to exploit parallelism automatically from a program have been made; however, the benefits of these have reached their
limit \cite[p. 7-12]{mccool_2012_structured_spppfec}. In order to fully utilize the increase in performance that multicore
architectures promise, programmers today must instead turn to explicit parallel programming.

Python is one of world's most popular programming languages \cite{krill_2015_python_psnhilp}. It is used extensively both at schools and
in the industry, and its benefits include expressiveness, portability, and the fact that it is easy to learn. Python has support for
parallel programming, although it has caveats and overheads associated with a concurrency-hampering mechanism called the
\emph{Global Interpreter Lock} \cite{beazley_150745UTC_introduction_aitpc}.

This thesis concerns a combination of the areas mentioned above: parallel computing using Python.

%\section{Dataset transfromation} \label{trioptima}
%The thesis is conducted at TriOptima, a company that provides different services for the OTC derivatives market.
%In one of TriOptima's services, customers upload datasets representing trades.
%OTC derivatives concern trading directly between two parties, and the customers include large banks. TriOptima’s services
%include portfolio compression, reconciliation, dispute resolution, and risk management. The services deal with substantial
%amounts of data, and face challenges such as high security demands, availability requirements, and speed optimization
%for data transformations and risk calculations. In their reconciliation and dispute resolution service,

\section{Motivation}
The motivation of this thesis is to answer the following questions.

\emph{Given the size of a dataset and its set of filters, is is possible to the determine 
if parallelization of the data transformation using Python will be beneficial or not?}

The thesis question gives rise to the following subquestions:
\begin{itemize}
    \item What is the best approach for parallelizing code in Python in order to minimize data races and maintain performance?
    \item How should the parallel performance be measured?
    \item What kind of data dependencies exist and how do they affect parallelization?
    \item What kind of overhead does parallelization introduce?
\end{itemize}

\section{Objective}
The objective of this thesis is to answer the questions stated above using a literature study and by implementing a working parallelization
of the existing dataset processing program, subsequently analysing transformations of several datasets in order to draw conclusions about performance.

\section{Contribution}
This thesis focuses on parallelization analysis of a file format rather than the more conventional method of analyzing source code. Additionally,
it shows how Python can be effectively used for parallelization in a complex system not built for parallelization from the start. The fact that
the parallelized system relies on database operations and, consequently, I/O is another aspect of the thesis that may interest other researchers
in the field of parallel programming. Similar projects can use the conclusions of this thesis as a foundation when creating a parallelization strategy.
