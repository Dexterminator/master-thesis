Purpose of chapter, TODO

\section{Technology}
In this section, technologies used in triResolve that will be mentioned throughout this chapter are briefly described.

\subsection{Django}
Django is a Python web development framework \cite{holovaty_chapter_c1itd}. It implements a version of the MVC (Model-View-Controller) pattern, which decouples request routing, data access, and
presentation. Django's model layer allows the programmer to retrieve and modify entities in an SQL database through Python code, without writing SQL.

\subsection{MySQL}
MySQL is an open source relational database system \cite{what_wim}. It is used by TriOptima as the database backend for Django.

\subsection{Cassandra}
Cassandra is a column-oriented \textit{NoSQL} database \cite[p. 1-9]{mishra_2014_beginning_bacd}. It features dynamic schemas, meaning that columns can be added to a schema as needed, and that
the number of columns may vary from row to row. Cassandra is designed to have no single point of failure, and instead uses a number of nodes in a peer-to-peer structure. This design is
employed in order to ensure high availability, with data replicated across the nodes. It uses the query language CQL (Cassandra Query Language), which is similar to relational database SQL.

\section{Trade files and data sets}
As mentioned briefly in the background section, users of the triResolve service upload \textit{trade files}, which contain one or several data sets with
rows of trade data such as party id, counterparty id, trade id, notional, and so on. An example of a trade data set (with some columns omitted) can be seen in figure
\ref{fig:data_set_example}.

\begin{figure}[ht]
\begin{tabular}{|c|c|c|p{3cm}|c|c|}%
  \hline
  \bfseries Party ID & \bfseries CP ID & \bfseries Trade ID & \bfseries Product class & \bfseries Trade curr & \bfseries Notional
  \csvreader[respect all,head to column names]{figures/EFET.csv}{PARTY_ID=\pid, CP_ID=\cpid, TRADE_ID=\tid, PRODUCT_CLASS=\pcls, TRADE_CURR=\tc, NOTIONAL=\notional}
  {\\\hline \pid & \cpid & \tid & \pcls & \tc & \notional}
  \\ \hline
\end{tabular}
  %\centering
\caption[Example of trade data set]{A simplified example of a trade data set uploaded by the users of triResolve.}
  \label{fig:data_set_example}
\end{figure}

\section{File formats and filters}
Different customers may have different ways of formatting their data sets, with different names for headers, varying column orders, extra fields,
and special rules. In order to convert these into a common format that make it possible to use the files in the same contexts, a file format specifying
how the data set in question should be processed is used. The format contains a set of \textit{filters} which should be applied to each row of the data set.

\subsection{Field mapping}

\section{Implementation}

\subsection{Analyzing filters for parallelizability}


\section{Evaluation}
