Purpose of chapter, TODO

\section{Technology}

\subsection{Django}

\subsection{MySQL}

\subsection{Cassandra}

\section{Trade files and data sets}
As mentioned briefly in the background section, users of the triResolve service upload \textit{trade files}, which contain one or several data sets with
rows of trade data such as party id, counterparty id, trade id, notional, and so on. An example of a trade data set (with some columns omitted) can be seen in figure
\ref{fig:data_set_example}.

\begin{figure}[ht]
\begin{tabular}{|c|c|c|p{3cm}|c|c|}%
  \hline
  \bfseries Party ID & \bfseries CP ID & \bfseries Trade ID & \bfseries Product class & \bfseries Trade curr & \bfseries Notional
  \csvreader[respect all,head to column names]{figures/EFET.csv}{PARTY_ID=\pid, CP_ID=\cpid, TRADE_ID=\tid, PRODUCT_CLASS=\pcls, TRADE_CURR=\tc, NOTIONAL=\notional}
  {\\\hline \pid & \cpid & \tid & \pcls & \tc & \notional}
  \\ \hline
\end{tabular}
  %\centering
\caption[Example of data set]{A simplified example of a trade data set uploaded by the users of triResolve.}
  \label{fig:data_set_example}
\end{figure}

\section{File formats and filters}
Different customers may have different ways of formatting their data sets, with different names for headers, varying column orders, extra fields,
and special rules. In order to convert these into a common format that make it possible to use the files in the same contexts, a file format specifying
how the data set in question should be processed is used. The format contains a set of \textit{filters} which should be applied to each row of the data set.

\subsection{Mapping}

\section{Implementation}

\subsection{Analyzing filters for parallelizability}


\section{Evaluation}
