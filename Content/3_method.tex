Purpose of chapter, TODO

\section{Technology}
In this section, technologies used in triResolve that will be mentioned throughout this chapter are briefly described.

\subsection{Django}
Django is a Python web development framework \cite{holovaty_chapter_c1itd}. It implements a version of the MVC (Model-View-Controller) pattern, which decouples request routing, data access, and
presentation. Django's model layer allows the programmer to retrieve and modify entities in an SQL database through Python code, without writing SQL.

\subsection{MySQL}
MySQL is an open source relational database system \cite{what_wim}. It is used by TriOptima as the database backend for Django.

\subsection{Cassandra}
Cassandra is a column-oriented \textit{NoSQL} database \cite[p. 1-9]{mishra_2014_beginning_bacd}. It features dynamic schemas, meaning that columns can be added to a schema as needed, and that
the number of columns may vary from row to row. Cassandra is designed to have no single point of failure, and instead uses a number of nodes in a peer-to-peer structure. This design is
employed in order to ensure high availability, with data replicated across the nodes. It uses the query language CQL (Cassandra Query Language), which is similar to relational database SQL.

\section{Trade files and data sets}
As mentioned briefly in the background section, users of the triResolve service upload \textit{trade files}, which contain one or several data sets with
rows of trade data such as party id, counterparty id, trade id, notional, and so on. An example of a trade data set (with some columns omitted) can be seen in figure
\ref{fig:data_set_example}.

\begin{figure}[ht]
\begin{tabular}{|c|c|c|p{3cm}|c|c|}%
  \hline
  \bfseries Party ID & \bfseries CP ID & \bfseries Trade ID & \bfseries Product class & \bfseries Trade curr & \bfseries Notional
  \csvreader[respect all,head to column names]{figures/EFET.csv}{PARTY_ID=\pid, CP_ID=\cpid, TRADE_ID=\tid, PRODUCT_CLASS=\pcls, TRADE_CURR=\tc, NOTIONAL=\notional}
  {\\\hline \pid & \cpid & \tid & \pcls & \tc & \notional}
  \\ \hline
\end{tabular}
  %\centering
\caption[Example of trade data set]{A simplified example of a trade data set uploaded by the users of triResolve.}
  \label{fig:data_set_example}
\end{figure}

\section{File formats}
Different customers may have different ways of formatting their data sets, with different names for headers, varying column orders, extra fields,
and special rules. In order to convert these into a common format that make it possible to use the files in the same contexts, a file format specifying
how the data set in question should be processed is used. The format contains a set of \textit{filters} which should be applied to each row of the data set.
The different filter configurations may affect how parallelizable the processing of data set is. The result of this processing is called a \textit{verification result},
and consists of rows in a Cassandra schema.

\section{Filters}
All filters used to create a verification result are outlined below.

\subsection{Mapping}
Maps a value from a column in the data set to a specified output column in the verification result. This is usually specified for each of the columns in the input
data set, and is the most common filter. The mappings may have small extra tuning attached to them, such as specifying a date format or extracting only part of the
text using regex. One of these extra tunings is attached to the trade id column, and is called \textit{Make unique}. This tuning keeps track of all trade id:s that
have been encountered so far, and, if it finds a duplicate, adds a suffix to it in order to ensure that all trade id:s are unique.

\subsection{Dataset translation}
A dataset translation is similar to a mapping, but uses specified columns in an external dataset to map input columns to output columns.

\subsection{Dataset information}
Extracts information about the dataset, such as the name or owner.

\subsection{Tradefile information}
Similar to the dataset information filter, except that it extracts information about the trade file that contains the data set.

\subsection{NullTranslationFilter}
In some data sets, other values than \code{NULL} are used to convey the absence of a value. This filter allows the user to specify which other values
should be interpreted as \code{NULL}.

\subsection{Relation currency}
If the currency that is supposed to be used in a relation (a party and a counterparty) is stored in the database and should be mapped to an output column, this
filter retrieves this information.

\subsection{Global variable}
A global variable filter writes a value to a variable that is accessed by subsequent filters on the same row, and by all filters on the rest of the rows in the data
set. A global variable can be written several times throughout the processing of a data set.

\subsection{State variable}
A state variable is similar to a global variable, but is always written to before all other processing of the data set begins.

\subsection{Temporary variable}
Similar to the other variables, except for the fact that it is only accessible during processing of the row where it was written. When the row has finished being
processed, the variable is cleared.

\subsection{Conditional block}
A conditional block works like the programming construct \code{if}. Performs a specified filter (which may also be a conditional block) only if a certain
condition is fulfilled. Most commonly, the condition take the form \code{field = value}, but may also involve more complex expressions in the form of a
subset of Python.

\subsection{Logger}
A logger filter simply logs the specified output. Can for instance be used when a user wants to know whenever a conditional block has been entered.

\subsection{Skip row}
Ignores the current row when processing. Usually used in a conditional block.

\subsection{Stop processing}
Stops processing the dataset, ignoring all subsequent rows. Can be used when the footer of the dataset contains information that should not be interpreted
as a trade.

\subsection{ThirdPartyAutoMapper}
When a customer has uploaded a trade file on behalf of another customer, this filter extracts the information needed to make sure that the data is loaded
for the correct customer.

\subsection{Set value}
Simply sets the value of the output column to the value that is entered.

\subsection{RegExp extract}
Extracts text from a column using regex, and writes matching groups to other columns

\subsection{RegExp replace}
Replaces column text matching some regex with a specified value.

\subsection{Analyzing filters for parallelizability}
Since the filters specify what the processing program should to to each row in a data set, they are the prime candidates for parallelization analysis.

\section{Implementation}

\section{Evaluation}
