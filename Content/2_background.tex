Purpose of chapter, TODO

\section{Definitions}
\begin{itemize}
  \item \emph{IPC} Interprocess communication.
  \item \emph{MPI} Message Passing Interface. Standardized interface for message passing between processes.
  \item \emph{Embarrassingly parallel} A problem that is embarrassingly parallel can easily be broken down into components that
    can be run in parallel. %Cite astro?
  \item \emph{Real time} The total time it takes for a call to finish.
  \item \emph{User time} The time a call takes, excluding system overhead; the time the call spends in user mode.
  \item \emph{System time} The time in a call that is consumed by system overhead; the time the call spends in kernel mode.
\end{itemize}

\section{Multicore architecture} %Find a better name?
\subsection{Processes vs threads}
while both threads and processes represent contexts in which a program is run, they have a few differences. A thread is run inside
a process, and the threads within the process share memory and state with each other and the parent
process \cite{singh_2013_parallel_padpwprfmm}. Individual processes do not share memory with each other, and any
communication between processes must be done with message passing rather than with shared memory. Consequently, communication
between threads is generally faster than between processes.
Typically, different threads can be scheduled on different cores, which is also true for different processes. %Fixa citation

\subsection{Scheduling}
Threads and processes are scheduled by the operating system, and the exact mechanism for choosing what to schedule when differs
between platforms and implementations \cite[p. 472]{herlihy_2012_art_taomprr}. Scheduling may imply running truly parallel
on different cores, or on the same core using time-slicing. Threads and processes may be descheduled from running temporarily for several
reasons, including issuing a time-consuming memory request.

\subsection{Multicore communication and caching}
Multiple processors communicate with each other through a bus or a network \cite[p. 472-476]{herlihy_2012_art_taomprr}. Since the
means of communication between the processes is a finite resource, too much traffic may result in delays. The processors typically
have their own cache. In order to avoid unnecessary reads from the slower main memory, processors may read from another processor
that has the requested data cached. In a process called \emph{cache coherence}, shared cached values are kept up to date using one
of several protocols. The effect that these different means of communication between processors has on performance in
multiprocessor programs should not be ignored.

\subsection{Data parallelism}
Data parallelism denotes code where the parallelism comes from decomposing the data and running it with the same piece of code
across several processors or computers \cite{singh_2013_parallel_padpwprfmm}. It allows scalability as number of cores and problem
sizes increase, since more parallelism can be exploited for larger data sets \cite[p. 24]{mccool_2012_structured_spppfec}.

\subsection{Task parallelism}
\section{Performance models for parallel speedup}
\subsection{Amdahl's law}
\subsection{Expanding Amdahl's law}
\subsection{Work-span model}
\section{Python performance and parallel capabilities}
\subsection{Performance}
\subsection{The GIL, Global Interpreter Lock}
\subsection{Threading}
\subsection{Multiprocessing}

\section{Related work}
\subsection{Efficient parallelization of path planning workload on single-chip shared-memory multicores}
Ahmad et al. \cite{ahmad_2015_efficient_epoppwossm} parallelize path planning algorithms such as Dijkstra's algorithm using C/C++ and
Python in order to compare the results and evaluate each language's suitability for parallel computing. For the Python implementation,
both the \code{multiprocessing} and \code{threading} packages are used. The authors identify Python as the preferable choice 
in application development, due to its safe nature in comparison to C and C++. The implementation using the \code{threading}
module resulted in no speedup over the sequential implementation. Parallelization using the \code{multithreading} module resulted
in a speedup of 2.5x for sparse graphs, and a speedup of 6.5x for dense graphs. The overhead introduced by the interpreted nature
of Python, as well as the extra costs associated with Python multiprocessing, was evident as the C/C++ implementations showed both
better performance and better scalability. The slowdowns for sparse graph of Python compared to C/C++ ranged between
20x to 700x depending on the graphs.
However, the authors note that the parallel Python implementation exhibits scalability in comparison to its sequential implementation.

The experiments were conducted on a machine with 4 cores with 2-way hyperthreading.


\subsection{Harnessing multicores: Strategies and implementations in ATLAS}
Binet et al. \cite{binet_2010_harnessing_hmsaiia} present a case study where parts of the ATLAS software used in
LHC (Large Hadron Collider) experiments are parallelized. Because of the complexity and sensitivity of the system,
one of the goals of the study is to minimize the code changes when implementing the parallelization. The authors highlight several
benefits of using multiple processes with IPC (interprocess communication) instead of traditional multithreading, including ease of 
implementation, explicit data sharing, and easier error recovery. The Python \code{multiprocessing} module was used to parallelize
the program, and the authors emphasize the decreased burden thanks to not having to implement explicit IPC and synchronization.
Finding the parts of the program that is embarrassingly parallel (DEFINE THIS) and parallelizing these is
identified as the preferred approach in order to avoid an undesirably large increase in complexity while
still producing a significant performance boost.

The parallel implementation was tested by measuring the user and real time (DEFINE THESE) for different numbers of processes.
These measurements show a clear increase in user time because of additional overhead, but also a steady decrease in real time.

\subsection{On the Performance of the Python Programming Language for Serial and Parallel Scientific Computations}
Cai et al. \cite{cai_2005_performance_otpotpplfsapsc} note that Python is suitable for scientific programming thanks to its richness and
power, as well as its interfacing capabilities with legacy software written in other languages. Among other experiments on Python
efficiency in scientific computing, its parallel capabilities are investigated. The Python MPI (DEFINE) package \code{Pypar} is used for
the parallelization, using typical MPI operations such as send and receive. The calculations, such as wave simulations, 
are made with the help of the \code{numpy} package for increased efficiency. The authors conclude that while communication 
introduces overhead, Python is sufficiently efficient for scientific parallel computing.

\subsection{Parallel astronomical data processing with Python: Recipes for multicore machines}
Singh et al. \cite{singh_2013_parallel_padpwprfmm} present Python as a fitting language for parallel computing, and use the
\code{multiprocessing} module as well as the standalone \code{Parallel Python} package in their experiments. Because of the
communication overhead in Python, the study focuses on embarrassingly parallel problems where little communication is needed.
Different means of parallelization are
compared: the Pool/Map approach, the Process Queue approach, and the Parallel Python approach. %TODO: Maybe define the approaches?
The results in general show significant time savings even though the approaches taken are relatively straightforward.
The best performance is achieved when the number of processes is equal to the number of physical cores on the computer.
The Process/Queue is shown to perform better than both Pool/Map and parallel Python. This comes at the cost of a slightly less
straightforward implementation. The impact of load balancing and chunk size is also discussed, with the conclusion that work load
should be evenly distributed among cores as computation is limited by the core that takes the longest to finish.

\subsection{Parallel optimal choropleth map classification in PySAL}
Rey et al. \cite{rey_2013_parallel_pocmcip} compare \code{multiprocessing} and \code{Parallel Python} with the GPU-based parallel
module \code{PyOpenCI} when attempting to parallelize portions of the spatial analysis library PySAL. In particular, different
versions of the Fisher-Jenks algorithm for classification are compared. For the smallest sample sizes, the overhead of the
different parallel implementations produce slower code, but as the sample sizes grow larger the speedup grow relatively quickly.
For the largest of the sample sizes, the speedup curve generally flattens out; the authors state this as counter-intuitive and
express an interest in investigating this further. In general, the CPU-based modules \code{multiprocessing} and \code{Parallel Python}
perform better than the GPU-based PyOpenCI. The \code{multiprocessing} module produced similar or better results than the
\code{Parallel Python} module.
While the parallel versions of the algorithm perform better, the bigger implementation effort associated with it is noted.

\subsection{PEP 0371}
In their proposal for the inclusion of the multiprocessing module into the Python standard library,
Noller and Oudkerk \cite{noller_pep_p0} include several benchmarks where the multiprocessing module's performance is compare to
that of the threading module. They emphasize the fact that the benchmarks are not as applicable on platforms with slow forking
time. The benchmarks show that while of course slower than sequential execution, multiprocessing performs better than
threading when just spawning workers and executing an empty function. For the CPU-bound task of computing Fibonacci numbers,
multiprocessing shows significantly better result than threading (which is in fact slower than sequential code). For I/O bound
calculations, which is an application considered suitable for the threading module, the multiprocessing module is still shown to have
the best performance when 4 or more workers are used.

The benchmarks where performed using the following hardware:
\begin{itemize}
  \item 4 Core Intel Xeon CPU @ 3.00GHz
  \item 16 GB of RAM
  \item Python 2.5.2 compiled on Gentoo Linux (kernel 2.6.18.6)
  \item pyProcessing 0.52
\end{itemize}

\subsection{Three Unique Implementations of Processes for PyCSP}
Friborg et al. \cite{friborg_2009_three_tuiopfp} explore the use of processes, threads and greenlets in their process abstraction
library PyCSP. The authors observe the clear performance benefits of using multiprocessing over threads due to the circumvention of the GIL
that the multiprocessing module allows. Greenlets are user-level threads that execute in the same thread and are unable to utilize
several cores. On Microsoft Windows, where the fork() system call is not available, the process creation is observed as
significantly slower than on UNIX-based platforms. While serialization and communication has a negative impact on performance when
using multiprocessing, the authors state that this produces the positive side-effect of processes not being able to
modify data received from other processes.

\subsection{Summary of related work}
Common themes and conclusions in the related work presented above include.

\begin{itemize}
  \item Python is a suitable language for parallel programming.
  \item The multiprocessing module is successful in circumventing the GIL often shows the same or better performance than other
    methods, even for I/O bound programs.
  \item The overhead that IPC introduces when creating parallel Python programs makes it imperative to minimize communication and
    synchronization. Consequently, embarrassingly parallel programs are preferable when using Python.
  \item For existing larger systems, extensive parallelization may produce undesired complexity.
\end{itemize}
