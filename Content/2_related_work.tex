\section{Related work} \label{related_work}
\subsection{Efficient parallelization of path planning workload on single-chip shared-memory multicores}
Ahmad et al. \cite{ahmad_2015_efficient_epoppwossm} parallelize path planning algorithms such as Dijkstra's algorithm using C/C++ and
Python in order to compare the results and evaluate each language's suitability for parallel computing. For the Python implementation,
both the \code{multiprocessing} and \code{threading} packages are used. The authors identify Python as the preferable choice 
in application development, due to its safe nature in comparison to C and C++. The implementation using the \code{threading}
module resulted in no speedup over the sequential implementation. Parallelization using the \code{multithreading} module resulted
in a speedup of 2.5x for sparse graphs, and a speedup of 6.5x for dense graphs. The overhead introduced by the interpreted nature
of Python, as well as the extra costs associated with Python multiprocessing, was evident as the C/C++ implementations showed both
better performance and better scalability. The slowdowns for sparse graph of Python compared to C/C++ ranged between
20x to 700x depending on the graphs.
However, the authors note that the parallel Python implementation exhibits scalability in comparison to its sequential implementation.

The experiments were conducted on a machine with 4 cores with 2-way hyperthreading.


\subsection{Harnessing multicores: Strategies and implementations in ATLAS}
Binet et al. \cite{binet_2010_harnessing_hmsaiia} present a case study where parts of the ATLAS software used in
LHC (Large Hadron Collider) experiments are parallelized. Because of the complexity and sensitivity of the system,
one of the goals of the study is to minimize the code changes when implementing the parallelization. The authors highlight several
benefits of using multiple processes with IPC instead of traditional multithreading, including ease of 
implementation, explicit data sharing, and easier error recovery. The Python \code{multiprocessing} module was used to parallelize
the program, and the authors emphasize the decreased burden resulting from not having to implement explicit IPC and synchronization.
Finding the parts of the program that are embarrassingly parallel and parallelizing these is
identified as the preferred approach in order to avoid an undesirably large increase in complexity while
still producing a significant performance boost.

The parallel implementation was tested by measuring the user and real time for different numbers of processes.
These measurements show a clear increase in user time because of additional overhead, but also a steady decrease in real time.

\subsection{On the Performance of the Python Programming Language for Serial and Parallel Scientific Computations}
Cai et al. \cite{cai_2005_performance_otpotpplfsapsc} note that Python is suitable for scientific programming thanks to its richness and
power, as well as its interfacing capabilities with legacy software written in other languages. Among other experiments on Python
efficiency in scientific computing, its parallel capabilities are investigated. The Python MPI package \code{Pypar} is used for
the parallelization, using typical MPI operations such as send and receive. The calculations, such as wave simulations, 
are made with the help of the \code{numpy} package for increased efficiency. The authors conclude that while communication 
introduces overhead, Python is sufficiently efficient for scientific parallel computing.

\subsection{Parallel astronomical data processing with Python: Recipes for multicore machines}
Singh et al. \cite{singh_2013_parallel_padpwprfmm} present Python as a fitting language for parallel computing, and use the
\code{multiprocessing} module as well as the standalone \code{Parallel Python} package in their experiments. Because of the
communication overhead in Python, the study focuses on embarrassingly parallel problems where little communication is needed.
Different means of parallelization are
compared: the Pool/Map approach, the Process/Queue approach, and the Parallel Python approach. 
In the Pool/Map approach, the simple functions of \code{multiprocessing.Pool} are used to specify a number of processes, a data
set, and the function to be executed with each element in the dataset as a parameter. In the Process/Queue approach, a
\code{multiprocessing.Queue} is spawned and filled with chunks of data. Several \code{multiprocessing.Process} objects are then
spawned, which all share the queue and get data to operate on from it while it is not empty. Another shared queue is used for
collecting the results. In the Parallel Python approach, the \code{Parallel Python} abstraction \emph{job server}
is used to submit tasks for each data chunk. The tasks are automatically executed in parallel by the job server, and the results
are collected when they have finished.

The results in general show significant time savings even though the approaches taken are relatively straightforward.
The best performance is achieved when the number of processes is equal to the number of physical cores on the computer.
The Process/Queue is shown to perform better than both Pool/Map and parallel Python. This comes at the cost of a slightly less
straightforward implementation. The impact of load balancing and chunk size is also discussed, with the conclusion that work load
should be evenly distributed among cores as computation is limited by the core that takes the longest to finish.

\subsection{Parallel optimal choropleth map classification in PySAL}
Rey et al. \cite{rey_2013_parallel_pocmcip} compare \code{multiprocessing} and \code{Parallel Python} with the GPU-based parallel
module \code{PyOpenCI} when attempting to parallelize portions of the spatial analysis library PySAL. In particular, different
versions of the Fisher-Jenks algorithm for classification are compared. For the smallest sample sizes, the overhead of the
different parallel implementations produce slower code, but as the sample sizes grow larger the speedup grows relatively quickly.
For the largest of the sample sizes, the speedup curve generally flattens out; the authors state this as counter-intuitive and
express an interest in investigating this further. In general, the CPU-based modules \code{multiprocessing} and \code{Parallel Python}
perform better than the GPU-based PyOpenCI. The \code{multiprocessing} module produced similar or better results than the
\code{Parallel Python} module.
While the parallel versions of the algorithm perform better, the bigger implementation effort associated with it is noted.

\subsection{PEP 0371}
In their proposal for the inclusion of the \code{multiprocessing} module into the Python standard library,
Noller and Oudkerk \cite{noller_pep_p0} include several benchmarks where the \code{multiprocessing} module's performance is
compared to
that of the \code{threading} module. They emphasize the fact that the benchmarks are not as applicable on platforms with slow forking
time. The benchmarks show that while naturally slower than sequential execution, \code{multiprocessing} performs better than
\code{threading} when simply spawning workers and executing an empty function. For the CPU-bound task of computing Fibonacci numbers,
\code{multiprocessing} shows significantly better result than \code{threading} (which is in fact slower than sequential code). For I/O bound
calculations, which is an application considered suitable for the \code{threading} module, the \code{multiprocessing} module is still shown to have
the best performance when 4 or more workers are used.

The benchmarks where performed using the following hardware:
\begin{itemize}
  \item 4 Core Intel Xeon CPU @ 3.00GHz
  \item 16 GB of RAM
  \item Python 2.5.2 compiled on Gentoo Linux (kernel 2.6.18.6)
  \item pyProcessing 0.52
\end{itemize}

\subsection{Three Unique Implementations of Processes for PyCSP}
Friborg et al. \cite{friborg_2009_three_tuiopfp} explore the use of processes, threads and greenlets in their process abstraction
library PyCSP. The authors observe the clear performance benefits of using multiprocessing over threads due to the circumvention of the GIL
that the \code{multiprocessing} module allows. Greenlets are user-level threads that execute in the same thread and are unable to utilize
several cores. On Microsoft Windows, where the fork() system call is not available, the process creation is observed as
significantly slower than on UNIX-based platforms. While serialization and communication has a negative impact on performance when
using \code{multiprocessing}, the authors state that this produces the positive side-effect of processes not being able to
modify data received from other processes.

\subsection{Summary of related work}
Common themes and conclusions in the related work presented above include:

\begin{itemize}
  \item Python is a suitable language for parallel programming.
  \item The \code{multiprocessing} module is successful in circumventing the GIL and consistently shows the same or better performance than other
    methods, even for I/O bound programs.
  \item The overhead that IPC introduces when creating parallel Python programs makes it imperative to minimize communication and
    synchronization. Consequently, embarrassingly parallel programs are preferable when using Python for parallelization.
  \item For existing larger systems, extensive parallelization may produce undesired complexity.
\end{itemize}
