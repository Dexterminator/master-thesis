\section{Parallel program profiler analysis}

\section{Test datasets}
For all datasets, the number of filters is limited to a few more than the number of columns, as the majority of the filters are column mapping filters.
This is typical for most datasets, making this a suitable sample of datasets. No datasets with inherently serial file formats where used, as these cannot be
parllelized and would provide no interesting data. Instead, the focus of the experiments are on the \textit{Extra overhead} file formats, since these are the
most common (since making trade ID:s unique is a useful feature). They also give the fairest indication of how successful the parallelization is, as they are
both the worst case (apart from inherently serial file formats) and the common case. Another reason why the set of datasets was chosen is their differing sizes,
with potentially differing parallelization benefits.

The characteristics of the test datasets are outlined below.

\subsection{Dataset 1}
\begin{itemize}
  \item \textbf{Rows:} 102
  \item \textbf{Columns:} 46
  \item \textbf{File format family:} Extra overhead
\end{itemize}

\subsection{Dataset 2}
\begin{itemize}
  \item \textbf{Rows:} 2,890
  \item \textbf{Columns:} 46
  \item \textbf{File format family:} Extra overhead
\end{itemize}

\subsection{Dataset 3}
\begin{itemize}
  \item \textbf{rows:} 23,763
  \item \textbf{columns:} 46
  \item \textbf{file format family:} Extra overhead
\end{itemize}

\subsection{Dataset 4}
\begin{itemize}
  \item \textbf{rows:} 338,730
  \item \textbf{columns:} 89
  \item \textbf{file format family:} Extra overhead
\end{itemize}

\subsection{Experiments}

