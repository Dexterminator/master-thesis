Finansiell data representeras ofta med rader av värden, samlade i en datamängd. Denna data måste transformeras
till ett standardformat för att möjliggöra jämförelser och matchning. Detta kan ta lång tid för stora
datamängder. Huvudmålet för detta examensarbete är att snabba upp dessa transformationer genom parallellisering
med hjälp av Python-modulen multiprocessing. Datamängderna omvandlas med hjälp av regler, kallade filter.
Dessa filter analyserades för att identifiera begränsningar på ordningen i vilken datamängden kan behandlas,
och därigenom finna en parallelliseringsstrategi. Python-profileraren cProfile användes även för att hitta
potentiella parallelliseringspunkter i koden. Denna analys resulterade i användandet av ett ``task''-baserat
tillvägagångssätt, där transformationen delades in i ett sekventiellt pre-processingsteg, ett parallelt steg 
där grupper av rader distribuerades ut bland arbetarprocesser, och ett sekventiellt post-processingsteg.

Implementationen testades genom transformation av fyra datamängder av olika storlekar, med upp till 16 
arbetarprocesser. Resultaten för de fyra datamängderna var en speedup på 0.5, 2.1, 3.8 respektive 4.81.
En linjär ökning i minnesanvändning uppvisades även. Experimenten resulterade i slutsatsen att
datamängdens storlek var en betydande faktor i hur mycket speedup som uppvisades, delvis på grund av faktumet
att de sekventiella delarna tar upp en mindre del av programmet. Den stora minnesåtgången noterades som
en nackdel med att använda multiprocessing i kombination med cachning, på grund av duplicerad data.

Detta examensarbete visar att det är möjligt att snabba upp datamängdstransformation genom att använda
radgrupper som tasks, även om en relativt låg speedup uppvisades.
